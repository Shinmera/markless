
\glossarystyle{altlist}
\makeglossaries

\defineglossary{encoding}{
  A particular interpretation of a sequence of bytes into distinguishable \gpl{character}.
}

\defineglossary{character}{
  A singular entity as specified by an \g{encoding}.
}

\defineglossary{alphanumeric}{
  Any \g{character} that is one of the following: \\
  \inline{0123456789}\\
  \inline{abcdefghijklmnopqrstuvwxyz}\\
  \inline{ABCDEFGHIJKLMNOPQRSTUVWXYZ}
}

\defineglossary{whitespace}{
  Any \g{character} that represents a horizontal gap. Examples include space, tab, zero-width space, etc.
}

\defineglossary{string}{
  A sequence of \gpl{character}.
}

\defineglossary{substring}{
  A sequence of \gpl{character} within a \g{string}.
}

\defineglossary{text}{
  Text is made up of a series of \gpl{string} and \gpl{textual component}.
}

\defineglossary{textual component}{
  A section of \g{text} with specific visual styling and representation properties.
}

\defineglossary{specified textual component}{
  A \g{textual component} that is declared in this specification.
}

\defineglossary{resulting textual component}{
  The \g{textual component} that the \g{directive} puts in place of the \g{identifier} in the \g{document}.
}

\defineglossary{style}{
  A \g{style} is an attribute of a \g{textual component} that specifies how the \g{textual component} and its contents are supposed to be visually represented in the \g{document}.
}

\defineglossary{document}{
1) The top-most \g{textual component} that is not contained in any other \g{textual component}.
2) A \g{string} to be interpreted into a \g{textual component} using rules outlined by \gpl{directive}.
}

\defineglossary{document format}{
  A set of grammar and semantics to \glink{interpretation}{interpret} the contents of a \g{document}.
}

\defineglossary{newline}{
  Any \g{character} that represents that a new line should be started.
}

\defineglossary{line}{
  Any sub-sequence within a \g{string} that is delimited by the \g{newline}. That is to say, a line always begins at either the beginning of the \g{string} or after the \g{newline}, and always ends at either the end of the \g{string} or with a \g{newline}.
}

\defineglossary{empty line}{
  A \g{line} that only contains \g{whitespace} and a \g{newline}, or a sole \g{newline}.
}

\defineglossary{match}{
  A \g{match} occurs if a \g{string} is exactly recognised by some specific pattern or method.
}

\defineglossary{identifier}{
  Some form of pattern or method by which a \g{string} is recognisable. More specifically, an \g{identifier} provides a means by which a \g{substring} can be distinguished from the rest of the \g{string}. 
}

\defineglossary{identifier specifier}{
  A pattern in \sectionref{identifier syntax} to specify the way in which the \g{identifier} can be recognised.
}

\defineglossary{directive}{
  A directive associates an \g{identifier} to a \g{textual component} and specifies what \g{text} to put into this \g{resulting textual component}.
}

\defineglossary{line directive}{
  A \g{directive} that spans one or more \gpl{line}.
}


\defineglossary{singular line directive}{
  A \g{line directive} as specified in \sectionref{singular line directives}.
}

\defineglossary{spanning line directive}{
  A \g{line directive} as specified in \sectionref{spanning line directives}.
}

\defineglossary{guarded line directive}{
  A \g{line directive} as specified in \sectionref{guarded line directives}.
}

\defineglossary{inline directive}{
  A \g{directive} that can appear at any point within a \g{string}.
}

\defineglossary{surrounding inline directive}{
  An \g{inline directive} as specified in \sectionref{surrounding inline directives}.
}

\defineglossary{entity inline directive}{
  An \g{inline directive} as specified in \sectionref{entity inline directives}.
}

\defineglossary{compound inline directive}{
  An \g{inline directive} as specified in \sectionref{compound inline directives}.
}

\defineglossary{equivalent}{
  Two objects are considered equivalent, if they denote the same meaning or idea. In specific, two \gpl{character} are equivalent, if they denote the same visual identity.
}

\defineglossary{interpretation}{
  The act of detecting \gpl{directive} and executing their effects on a \g{document}.
}

\defineglossary{implementation}{
  Some form of program or system that implements the semantics of Markless.
}

\defineglossary{conforming implementation}{
  An \g{implementation} that fully and correctly adheres to all requirements laid down by this specification. An \g{implementation} may support additional features not described in this specification and still be conforming, as long as none of the features interfere with the \g{interpretation} of a \g{conforming document}.
}

\defineglossary{conforming document}{
  A \g{document} that does not violate any of the requirements set forth by the \gpl{directive} outlined in this specification and can thus be properly \glink{interpretation}{interpreted} by any \g{conforming implementation}.
}

\defineglossary{level}{
  A number representing the depth of a \g{directive} within the \g{document}. The level within any \g{directive} is one higher than the level the \g{directive} itself is at. The level of the \g{document} is always 0.
}

\defineglossary{binding}{
  A \syntax{binding} syntax rule, the content of which is the \g{string} it \glink{match}{matches}.
}

\defineglossary{content binding}{
  A \g{binding} with the \syntax{name} \inline$content$.
}

%%% Local Variables:
%%% mode: latex
%%% TeX-master: "0-markless"
%%% TeX-engine: luatex
%%% End:
