
\glossarystyle{altlist}
\makeglossaries

\defineglossary{encoding}{
  A particular interpretation of a sequence of bytes into distinguishable \gpl{character}.
}

\defineglossary{character}{
  A singular entity as specified by an \g{encoding}.
}

\defineglossary{string}{
  A sequence of \gpl{character}.
}

\defineglossary{text}{
  Text is made up of a series of \gpl{string} and \gpl{textual component}.
}

\defineglossary{textual component}{
  A section of \g{text} with specific visual styling and representation properties.
}

\defineglossary{document}{
1) The top-most \g{textual component} that is not contained in any other \g{textual component}.
2) A \g{string} to be interpreted into a \g{textual component} using rules outlined by \gpl{directive}.
}

\defineglossary{document format}{
  A set of grammar and semantics to \glink{interpretation}{interpret} the contents of a \g{document}.
}

\defineglossary{newline}{
  Any \g{character} that represents that a new line should be started.
}

\defineglossary{line}{
  Any sub-sequence within a \g{string} that is delimited by the \g{newline}. That is to say, a line always begins at either the beginning of the \g{string} or after the \g{newline}, and always ends at either the end of the \g{string} or with a \g{newline}.
}

\defineglossary{directive}{
  A distinctive sequence of \gpl{character} that marks a region within a \g{string} to belong to a specified \g{textual component}.
}

\defineglossary{line directive}{
  A \g{directive} that spans one or more \gpl{line}.
}

\defineglossary{equivalent}{
  Two objects are considered equivalent, if they denote the same meaning or idea. In specific, two \gpl{character} are equivalent, if they denote the same visual identity.
}

\defineglossary{interpretation}{
  The act of detecting \gpl{directive} and executing their effects on a \g{document}.
}

\defineglossary{implementation}{
  Some form of program or system that implements the semantics of Markless.
}

\defineglossary{conforming implementation}{
  An \g{implementation} that fully and correctly adheres to all requirements laid down by this specification. An \g{implementation} may support additional features not described in this specification and still be conforming, as long as none of the features interfere with the \g{interpretation} of a \g{conforming document}.
}

\defineglossary{conforming document}{
  A \g{document} that does not violate any of the requirements set forth by the \gpl{directive} outlined in this specification and can thus be properly \gpl{interpretation}{interpreted} by any \g{conforming implementation}.
}

%%% Local Variables:
%%% mode: latex
%%% TeX-master: "0-markless"
%%% End:
