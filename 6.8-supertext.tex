\definesubsection{Supertext}
\defineidentifier{supertext}{^<start [(]+><content ![)].*><end [)]+>}
\definetextualcomponent{supertext}{vertical-align: super} \\

Supertext is a \g{surrounding inline directive}. It marks the \g{text} to belong to a \g{textual component} that sets the style of the text to appear smaller and above the default text line. Only the \g{text} held by the \g{content binding} is outputted to the \g{resulting textual component}. The length of the \g{text} held by the \inline$start$ and \inline$end$ \gpl{binding} must be the same in order for the \ident{supertext} \g{identifier} to \g{match}.\\

\begin{examples}
  \example{This is a good example ^([citation needed])}{This is a good example \raisebox{.4ex}{\scriptsize [citation needed]}}
  \example{Nesting ^((supertext ^(is silly)))}{Nesting \raisebox{.4ex}{\scriptsize supertext \raisebox{.4ex}{\tiny is silly}}}
\end{examples}

%%% Local Variables:
%%% mode: latex
%%% TeX-master: "markless"
%%% TeX-engine: luatex
%%% TeX-command-extra-options: "-shell-escape"
%%% End: