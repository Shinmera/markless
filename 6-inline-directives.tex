\definesection{Inline Directives}
A \g{directive} is an \g{inline directive} if its identification is not bound to \gpl{line}. Unlike \gpl{line directive} therefore it can potentially be identified at any point in a string and span any length. \\

Any \g{textual component} specified by an \g{inline directive} can only contain \gpl{textual component} specified by \gpl{inline directive}. Furthermore, an \g{inline directive} cannot contain another \g{inline directive} of its own type at any \g{level}. An \g{inline directive} may further restrict which \gpl{directive} may appear within itself. An \g{inline directive} cannot cross the boundaries of another \g{directive} of a different kind. If such a case were to occur, the current \g{inline directive} is forcibly ended without regard for any possible trailing \g{match}. A special exception is made in the case of \gpl{spanning line directive}: since a \g{spanning line directive} is the combination of multiple matches of the same kind on consecutive lines into a singular \g{textual component}, an \g{inline directive} must be allowed to span over multiple matches.

\definesubsubsection{Surrounding Inline Directives}
An \g{inline directive} is a \g{surrounding inline directive} if its \g{identifier} is terminated by a \syntax{binding-reference} to a \syntax{binding} at the start of the \g{identifier} or its \g{identifier} contains a \inline$start$ and \inline$end$ \syntax{binding}, both of which are required to be the \g{equivalent} in length. The \g{resulting textual component} of a \g{surrounding inline directive} never contains the \gpl{string} \glink{match}{matched} by the \g{identifier}'s starting \syntax{binding}. Instead, each \g{surrounding inline directive} \g{identifier} contains a \g{content binding} that matches all the \g{text} that the \g{resulting textual component} will contain. \\

When a \g{surrounding inline directive} is \glink{processing}{processed}, processing begins anew over the \g{content binding} until the the part of the \g{identifier} after the \g{content binding} is \glink{full match}{fully matched}, at which point the \g{resulting textual component} is ended and control is handed back to the \g{standard processing loop}.

\definesubsubsection{Entity Inline Directives}
An \g{inline directive} is an \g{entity inline directive} if its \g{identifier} does not contain any \syntax{binding}s and instead the \g{text} of the \g{resulting textual component} is entirely dependant on the \g{entity inline directive} specification. \\

When a \g{entity inline directive} is \glink{processing}{processed}, the \g{resulting textual component} is ended once the \g{identifier} has been \glink{full match}{fully matched}. Then control is handed back to the \g{standard processing loop}.

\definesubsubsection{Compound Inline Directives}
An \g{inline directive} is a \g{compound inline directive} if its \g{identifier} consists of multiple \syntax{binding}s the contents of which are in some form outputted to the \g{resulting textual component}. \\

\definesubsection{Bold}
\defineidentifier{bold}{\*<content .*>\*}
\definetextualcomponent{bold}{font-weight: bold} \\

The bold \g{directive} is a \g{surrounding inline directive} that marks the \g{text} to belong to a \g{textual component} that sets the weight of the font to bold. Only the \g{text} held by the \g{content binding} is outputted to the \g{resulting textual component}. \\

\begin{examples}
  \example{not *bold* at all}{not \textbf{bold} at all}
  \example{and *some \\*things\\* are bad*}{and \textbf{some *things* are bad}}
\end{examples}

%%% Local Variables:
%%% mode: latex
%%% TeX-master: "markless"
%%% TeX-engine: luatex
%%% TeX-command-extra-options: "-shell-escape"
%%% End:

\definesubsection{Italic}
\defineidentifier{italic}{/<content .*>/}
\definetextualcomponent{italic}{font-style: italic} \\

Italic is a \g{surrounding inline directive}. It marks the \g{text} to belong to a \g{textual component} that sets the style of the font to italic. Only the \g{text} held by the \g{content binding} is outputted to the \g{resulting textual component}. \\

\begin{examples}
  \example{I /really/ don't care.}{I \textit{really} don't care.}
  \example{/call\/cc/ is important.}{\textit{call/cc} is important.}
\end{examples}

%%% Local Variables:
%%% mode: latex
%%% TeX-master: "markless"
%%% TeX-engine: luatex
%%% TeX-command-extra-options: "-shell-escape"
%%% End:

\definesubsection{Underline}
\defineidentifier{underline}{_<content .*>_}
\definetextualcomponent{underline}{text-decoration: underline} \\

Underline is a \g{surrounding inline directive}. It marks the \g{text} to belong to a \g{textual component} that sets the style of the text to underline. Only the \g{text} held by the \g{content binding} is outputted to the \g{resulting textual component}. \\

\begin{examples}
  \example{We _must_ finish this.}{We \underline{must} finish this.}
  \example{This _CONSTANT\_VALUE_ is variable.}{This \underline{CONSTANT\_VALUE} is variable.}
\end{examples}

%%% Local Variables:
%%% mode: latex
%%% TeX-master: "markless"
%%% TeX-engine: luatex
%%% TeX-command-extra-options: "-shell-escape"
%%% End:

\definesubsection{Strikethrough}
\defineidentifier{strikethrough}{\\<-<content .*>-\\>}
\definetextualcomponent{strikethrough}{text-decoration: strikethrough} \\

Strikethrough is a \g{surrounding inline directive}. It marks the \g{text} to belong to a \g{textual component} that sets the style of the text to strikethrough. Only the \g{text} held by the \g{content binding} is outputted to the \g{resulting textual component}.\\

\begin{examples}
  \example{To Do: <-nothing->}{To Do: \sout{nothing}}
  \example{<-Solve LOAD-TIME-VALUE problem->}{\sout{Solve LOAD-TIME-VALUE problem}}
  \example{<-Go -\\> there->}{\sout{Go -> there}}
\end{examples}

%%% Local Variables:
%%% mode: latex
%%% TeX-master: "markless"
%%% TeX-engine: luatex
%%% TeX-command-extra-options: "-shell-escape"
%%% End:

\definesubsection{Code}
\defineidentifier{code}{<start [`]+><content !<start>.*><start>}
\definetextualcomponent{code}{font-family: monospace} \\

Code is a \g{surrounding inline directive}. It marks the \g{text} to belong to a \g{textual component} that sets the font-family to monospace. Only the \g{text} held by the \g{content binding} is outputted to the \g{resulting textual component}. The code \g{directive} cannot contain any other \gpl{directive}. \\

\begin{examples}
  \example{Call `compile`}{Call \inline$compile$}
  \example{Earmuffs `*around*` your specials.}{Earmuffs \inline$*around*$ your specials.}
\end{examples}

%%% Local Variables:
%%% mode: latex
%%% TeX-master: "0-markless"
%%% TeX-engine: luatex
%%% End:

\definesubsection{Dashes}
\defineidentifier{en-dash}{--}
\defineidentifier{em-dash}{---}
\definetextualcomponent{en-dash}{display: en-dash}
\definetextualcomponent{em-dash}{display: em-dash} \\

The dashes are \gpl{entity inline directive}. If the \g{document} does not have direct support for dashes, a fallback character may be used when appropriate instead. In unicode encoded documents, this should be \unicode{2013} for the en-dash and \unicode{2014} for the em-dash. \\

\begin{examples}
  \example{A game -- or gamble --- if you will.}{A game -- or gamble --- if you will.}
\end{examples}

%%% Local Variables:
%%% mode: latex
%%% TeX-master: "markless"
%%% TeX-engine: luatex
%%% TeX-command-extra-options: "-shell-escape"
%%% End:

\definesubsection{Subtext}
\defineidentifier{subtext}{v<start [(]+><content ![()].*><end [)]+>}
\definetextualcomponent{subtext}{vertical-align: sub} \\

Subtext is an \g{inline directive}. It marks the \g{text} to belong to a \g{textual component} that sets the style of the text to appear smaller and below the default text line. Only the \g{text} held by the \g{content binding} is outputted to the \g{resulting textual component}. The length of the \g{text} held by the \inline$start$ and \inline$end$ \gpl{binding} must be the same in order for the \ident{subtext} \g{identifier} to \g{match}.\\

\begin{examples}
  \example{This is an example v(just so you know)}{This is an example \raisebox{-.4ex}{\scriptsize just so you know}}
  \longexample{Sometimes you have to be discreet v((or so they say (I wouldn't know))).}{Sometimes you have to be discreet \raisebox{-.4ex}{\scriptsize or so they say (I wouldn't know)}.}
\end{examples}

%%% Local Variables:
%%% mode: latex
%%% TeX-master: "0-markless"
%%% TeX-engine: luatex
%%% End:
\definesubsection{Supertext}
\defineidentifier{supertext}{^<start [(]+><content ![)].*><end [)]+>}
\definetextualcomponent{supertext}{vertical-align: super} \\

Supertext is a \g{surrounding inline directive}. It marks the \g{text} to belong to a \g{textual component} that sets the style of the text to appear smaller and above the default text line. Only the \g{text} held by the \g{content binding} is outputted to the \g{resulting textual component}. The length of the \g{text} held by the \inline$start$ and \inline$end$ \gpl{binding} must be the same in order for the \ident{supertext} \g{identifier} to \g{match}.\\

\begin{examples}
  \example{This is a good example ^([citation needed])}{This is a good example \raisebox{.4ex}{\scriptsize [citation needed]}}
  \example{Nesting ^((supertext ^(is silly)))}{Nesting \raisebox{.4ex}{\scriptsize supertext \raisebox{.4ex}{\tiny is silly}}}
\end{examples}

%%% Local Variables:
%%% mode: latex
%%% TeX-master: "markless"
%%% TeX-engine: luatex
%%% TeX-command-extra-options: "-shell-escape"
%%% End:
\definesubsection{URL}
\defineidentifier{url}{<target ~a(~w|[+-.])*://(~w|[\$-_.+!*'()\&+,/:;=?@\%])+>}
\definetextualcomponent{url}{interaction: link; target: target} \\

URL is an \g{inline directive} that marks the \g{text} to belong to a \g{textual component} that sets its interaction to allow following to the URL target. The user must be presented with an action that allows them to follow to the URL target. The exact manner in which the target is followed as well as the way in which the action is presented are \g{implementation dependant}. The \g{text} of the \g{resulting textual component} must be exactly the same as that of the \inline$target$ \g{binding}.\\

\begin{examples}
  \longexample{Come chat with us at irc://irc.freenode.net/\%23shirakumo !}{Come chat with us at \url{irc://irc.freenode.net/\%23shirakumo} !}
\end{examples}

%%% Local Variables:
%%% mode: latex
%%% TeX-master: "markless"
%%% TeX-engine: luatex
%%% TeX-command-extra-options: "-shell-escape"
%%% End:
\definesubsection{Compound}
\defineidentifier{compound}{"<content .>"\\(<option .*>(, *<option .*>)*\\)}
\definetextualcomponent{compound}{} \\

The compound \g{directive} is a \g{compound inline directive}. It determines its \g{style} dynamically by the additive combination of present options in the \inline$option$ binding. In the case where the style combination of two options conflicts, the style of the last option has priority. \\

Only the \g{text} held by the \g{content binding} is outputted to the \g{resulting textual component}. The \inline$option$ binding cannot contain any other \gpl{directive}.\\

An \g{implementation} must at least support the options specified in this section, but may add additional options the syntax and implications of which are completely \g{implementation dependant}. If an option is found that the \g{implementation} does not support or recognise, it is ignored and a \g{warning} may be \glink{signalling}{signalled}. \\

When a compound \g{directive} is \glink{processing}{processed}, processing begins anew over the \g{content binding}. Once the \inline$options$ binding has been \glink{full match}{fully matched}, the \g{resulting textual component} is ended and control is handed back to the \g{standard processing loop}.

\definesubsubsection{Bold}
\defineidentifier{compound-bold}{bold}
\definestyle{compound-bold}{font-weight: bold} \\

If given, this option marks the \g{style} to bold the \g{text}.\\

\begin{examples}
  \example{Not "again"(in bold)!}{Not \textbf{again}!}
\end{examples}

%%% Local Variables:
%%% mode: latex
%%% TeX-master: "markless"
%%% TeX-engine: luatex
%%% TeX-command-extra-options: "-shell-escape"
%%% End:

\definesubsubsection{Italic}
\defineidentifier{compound-italic}{italic}
\definestyle{compound-italic}{font-style: italic} \\

If given, this option marks the \g{style} to italicise the \g{text}.\\

\begin{examples}
  \example{This is "really"(in italic) important!}{This is \textit{really} important!}
\end{examples}

%%% Local Variables:
%%% mode: latex
%%% TeX-master: "markless"
%%% TeX-engine: luatex
%%% TeX-command-extra-options: "-shell-escape"
%%% End:

\definesubsubsection{Underline}
\defineidentifier{compound-underline}{underline}
\definestyle{compound-underline}{text-decoration: underline} \\

If given, this option marks the \g{style} to be set to underline the \g{text}.\\

\begin{examples}
  \example{Solve it "today"(in underline)!}{Solve it \underline{today}!}
\end{examples}

%%% Local Variables:
%%% mode: latex
%%% TeX-master: "markless"
%%% TeX-engine: luatex
%%% TeX-command-extra-options: "-shell-escape"
%%% End:

\definesubsubsection{Strikethrough}
\defineidentifier{compound-strikethrough}{strikethrough}
\definestyle{compound-strikethrough}{text-decoration: strikethrough} \\

If given, this option marks the \g{style} to be set to strikethrough the \g{text}.\\

\begin{examples}
  \example{"This is a good idea"(strikethrough).}{\sout{This is a good idea}.}
\end{examples}

%%% Local Variables:
%%% mode: latex
%%% TeX-master: "markless"
%%% TeX-engine: luatex
%%% TeX-command-extra-options: "-shell-escape"
%%% End:

\definesubsubsection{Spoiler}
\defineidentifier{compound-spoiler}{spoiler}
\definestyle{compound-spoiler}{display: hidden} \\

If given, this option marks the \g{style} to obscure the \g{text} in such a manner that the \g{user} must perform an \g{action} in order to reveal the \g{text}.\\

\begin{examples}
  \example{This is a secret(in spoiler)!}{This is a \colorbox{black}{secret}!}
\end{examples}

%%% Local Variables:
%%% mode: latex
%%% TeX-master: "0-markless"
%%% TeX-engine: luatex
%%% End:

\definesubsubsection{Font}
\defineidentifier{compound-font}{font <font .+>}
\definestyle{compound-font}{font-family: font} \\

If given, this option marks the \g{style} to change the font family. If the specified font is not available to the \g{user} for one reason or another, no font change occurs. The \g{implementation} may make an effort to include the font in the \g{document} in such a way that it is not necessary for the user to have a copy of the font, but it is not required to. \\

\begin{examples}
  \longexample{"Comic sans"(in font Comic Sans Ms) is a good font to annoy people.}{{\fontspec{Comic Sans MS}\selectfont Comic sans} is a good font to annoy people.}
\end{examples}

%%% Local Variables:
%%% mode: latex
%%% TeX-master: "markless"
%%% TeX-engine: luatex
%%% End:

\definesubsubsection{Color}
\defineidentifier{compound-color}{(color (<hex #.+>|<r ~n+>,<g ~n+>,<b ~n+>))|<name .+>}
\definestyle{compound-color}{color: color} \\

If given, this option marks the \g{style} to change the colour. The colour can be given in three ways:
\begin{enumerate}
\item Through a hexadecimal notation, contained in the \inline$hex$ \g{binding}. The \g{hexadecimal number} following the \inline$#$ must be exactly six \gpl{character} long.
\item Through a red, green, blue component notation, contained in the \inline$r$,\inline$g$, and \inline$b$ \gpl{binding}. Each of these bindings must contain a \g{decimal number} that may only range between 0 and 255. If the number lies outside this range, it is clamped to the nearest boundary.
\item Through an explicit colour name, contained in the \inline$name$ \g{binding}. The name must be \g{case insensitive}. The set of supported colour names is \g{implementation dependant}.
\end{enumerate}
If the specified colour value is invalid or unknown to the \g{implementation} according to the above restrictions, an \g{error} is \glink{signalling}{signalled}. If the \g{document} does not support the specified colour, the \g{implementation} must choose an alternative colour that approximates the specified one as closely as possible. \\

\definecolor{5-10-7-2}{RGB}{157,14,204}
\definecolor{5-10-7-3}{RGB}{145,16,16}
\begin{examples}
  \example{This is blue(in blue).}{This is \textcolor{blue}{blue}.}
  \example{Magic!(in color #9D0ECC)}{\textcolor{5-10-7-2}{Magic!}}
  \example{Now in technicolor(in color 145,16,16).}{Now in \textcolor{5-10-7-3}{technicolor}.}
\end{examples}

%%% Local Variables:
%%% mode: latex
%%% TeX-master: "markless"
%%% TeX-engine: luatex
%%% End:

\definesubsubsection{Size}
\defineidentifier{compound-size}{(size )?(<point ~n+pt>|<em ~n+?(\\.~n+?)?em>|<name .+>)}
\definestyle{compound-size}{font-size: size} \\

This option marks the \g{style} to change the font size. The size can be given in three ways:
\begin{enumerate}
\item Through a point value, contained in the \inline$point$ \g{binding}. The \g{real number} must be greater than zero.
\item Through an em value, contained in the \inline$em$ \g{binding}. The \g{real number} must be greater than zero. The font size is scaled according to the \g{real number} multiplied by the font size of the \g{textual component} one \g{level} below.
\item Through a name, contained in the \inline$name$ \g{binding}. The name must be \g{case insensitive}. At least the following names, corresponding to scaling factors, must be supported by the \g{implementation}:
  \begin{itemize}[noitemsep]
  \item Microscopic 0.25em
  \item Tiny 0.5em
  \item Small 0.8em
  \item Normal 1.0em
  \item Big 1.5em
  \item Large 2.0em
  \item Huge 2.5em
  \item Gigantic 4.0em
  \end{itemize}
  An implementation may support additional names, the exact sizing effects of which are \g{implementation dependant}.
\end{enumerate}
If the specified size value is invalid or unknown to the \g{implementation} according to the above restrictions, no size change occurs. \\

\begin{examples}
  \example{Oh "shit!"(in huge)}{Oh {\fontsize{2.5em}{2.6em}\selectfont shit!}}
  \example{In "20pt."(in 20pt)}{In {\fontsize{20pt}{20pt}\selectfont 20pt.}}
  \longexample{Well ""uh, "I don't know..."(in size 0.5em)""(in size 0.8em)}{Well {\fontsize{0.8em}{0.8em}\selectfont uh, {\fontsize{0.5em}{0.5em}\selectfont I don't know...}}}
\end{examples}

%%% Local Variables:
%%% mode: latex
%%% TeX-master: "0-markless"
%%% TeX-engine: luatex
%%% End:

\definesubsubsection{Hyperlink}
\defineidentifier{compound-hyperlink}{\{url\}|(#<internal .+>)|(link <external .+>)}
\definestyle{compound-hyperlink}{interaction: link;target: target} \\

This option marks the \g{style} to set the interaction to allow following to the target. The user must be presented with an action that allows them to follow to the target. The exact manner in which the target is followed as well as the way in which the action is presented are \g{implementation dependant}. The target can be given in three ways:
\begin{enumerate}
\item As an URL, contained in the \inline$target$ \g{binding}. In this case the semantics are the same as for the \comp{URL} \g{textual component}.
\item As an external reference, contained in the \inline$external$ \g{binding}. The exact semantics and allowed values for external references are \g{implementation dependant}.
\item As an internal reference, contained in the \inline$internal$ \g{binding}. The target is set to the position of the \g{textual component} associated with the \g{label} of the same name as the contents of the \g{binding}.
\end{enumerate}
If the specified target is invalid or unknown to the \g{implementation} according to the above restrictions, no interaction change occurs. \\

\begin{examples}
  \longexample{The "hyperspec"(to http://l1sp.org/cl/) is very useful.}{The \href{http://l1sp.org/cl/}{hyperspec} is very useful.}
  \example{And in "part 2"(to #identifier-syntax)...}{And in \hyperref[section:IDENTIFIER SYNTAX]{part 2}...}
  \example{I drew "something"(to ~/drawings/test.jpg) today.}{I drew \href{run:~/drawings/test.jpg}{something} today.}
\end{examples}

%%% Local Variables:
%%% mode: latex
%%% TeX-master: "markless"
%%% TeX-engine: luatex
%%% TeX-command-extra-options: "-shell-escape"
%%% End:


%%% Local Variables:
%%% mode: latex
%%% TeX-master: "markless"
%%% TeX-engine: luatex
%%% TeX-command-extra-options: "-shell-escape"
%%% End:

%%% Local Variables:
%%% mode: latex
%%% TeX-master: "markless"
%%% TeX-engine: luatex
%%% End: