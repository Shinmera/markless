\definesubsubsubsection{Line Break Mode}
\definevariable{line-break-mode}{show} \\

The \var{line-break-mode} variable may only assume two values: \inline$show$, and \inline$hide$. If the line break mode is \inline$show$, when the processor encounters an unescaped \g{newline}, a new \g{line} is started in the output \g{document}. \\

\begin{examples}
\begin{examplesource}
! set line-break-mode show
foo
bar\
baz
! set line-break-mode hide
bada
boom
\end{examplesource}
  \begin{exampleoutput}
    foo\\
    barbaz\\
    badaboom
  \end{exampleoutput}
\end{examples}

%%% Local Variables:
%%% mode: latex
%%% TeX-master: "markless"
%%% TeX-engine: luatex
%%% TeX-command-extra-options: "-shell-escape"
%%% End: