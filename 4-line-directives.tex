\definesection{Line Directives}
In order for a \g{directive} to be a \g{line directive}, its \g{identifier} must \g{match} the beginning of a \g{line} and either the end of a \g{line} or the beginning of a different \g{line}. Thus the \g{identifier} of each \g{line directive} only matches at the beginning of a \g{line}.\\

A \g{textual component} specified by a \g{line directive} can potentially contain any other \g{textual component}. Therefore, any \g{directive} is potentially recognisable within a \g{line directive}, including other \gpl{line directive}. However, a \g{line directive} may explicitly restrict which \gpl{directive} are recognised within itself. A \g{line directive} cannot cross the boundaries of another \g{line directive} of a different kind. If such a case were to occur, the current \g{line directive} is forcibly ended without regard for any possible trailing \g{match}.

\definesubsubsection{Singular Line Directives}
A \g{line directive} is a \g{singular line directive} if it is only ever active for a single \g{line}. If it is matched on two consecutive \gpl{line} this results in two separate \gpl{resulting textual component}.

\definesubsubsection{Spanning Line Directives}
A \g{line directive} is a \g{spanning line directive} if the \g{identifier} contains a \g{content binding}, and if \glink{match}{matches} on consecutive \gpl{line} of the \g{identifier} are interpreted as a single \g{match}. The semantics of such a spanning match are as follows: Only a single \g{resulting textual component} is produced for all the consecutively \glink{match}{matching} \gpl{line}. The \g{text} of this \g{resulting textual component} is produced by concatenating the contents of the \g{content binding} on each \g{line}. If the \g{content binding} does not \g{match} the \g{newline} on every \g{line}, the \g{newline} must be inserted between each \g{string} of the \g{content binding}.

\definesubsubsection{Guarded Line Directives}
A \g{line directive} is a \g{guarded line directive} if its \glink{match}{matched} region is specified by two \gpl{identifier} that each match a single \g{line}. The \g{text} of the \g{resulting textual component} is the \g{text} from the \g{line} immediately after the \g{line} the first \g{identifier} \glink{match}{matches} until and including the \g{line} immediately before the \g{line} the second \g{identifier} \glink{match}{matches}.

\definesubsection{Paragraph}
\begin{identifier}{paragraph}
<spaces [ ]*><content![ ].*>
\end{identifier}
\definetextualcomponent{paragraph}{margin: top, bottom} \\

The paragraph is the default \g{textual component} and acts as a fall-back. \Gpl{line} belong to the same paragraph until the length of \inline{spaces} changes, a new \g{inline directive} is recognised, or an \g{empty line} is encountered. The paragraph is a \g{spanning line directive}. The paragraph \g{directive} can only contain \gpl{inline directive}. \\

Paragraphs are visually distinguished by a margin above and below the \g{text}. An \g{implementation} may additionally employ indentation rules to distinguish the beginning of a paragraph. \\

\begin{examples}
  \begin{examplesource}
This is a paragraph
that spans multiple lines

This is another paragraph.
  \end{examplesource}
  \begin{exampleoutput}
    This is a paragraph\\
    that spans multiple lines.\\
    \\
    This is another paragraph.
  \end{exampleoutput}
  \begin{examplesource}
Paragraph One
  Paragraph Two
  \end{examplesource}
  \begin{exampleoutput}
    Paragaph One\\
    \\
    Paragraph Two
  \end{exampleoutput}
\end{examples}

%%% Local Variables:
%%% mode: latex
%%% TeX-master: "0-markless"
%%% TeX-engine: luatex
%%% End:

\definesubsection{Blockquote}
\begin{identifier}{blockquote header}
\~ <content .+>
\end{identifier}
\begin{identifier}{blockquote body}
| <content .*>
\end{identifier}
\definetextualcomponent{blockquote header}{margin: left; font-weight: bold}
\definetextualcomponent{blockquote body}{margin: left} \\

The blockquote header is a \g{singular line directive} that identifies the source of a quote. Only the \g{text} held by the \g{content binding} is outputted into the \g{resulting textual component}. The blockquote header can only contain \gpl{inline directive}. \\

The blockquote body is a \g{spanning line directive} that identifies a body of \g{text} that is being quoted. The blockquote body can contain any \g{directive} with the condition that the \gpl{directive} are matched against the \g{text} of the \g{resulting textual component}. \\

An implementation may choose to group the \comp{blockquote header} and \comp{blockquote body} together and reorder them if they are found consecutive to one another. However, a body can only ever be grouped together with a single header. In the case where a header lies between two bodies, the header is counted to belong to the second body. If a header is found without a corresponding body, the \g{implementation} may \glink{signalling}{signal} a \g{warning}. \\

\begin{examples}
  \begin{examplesource}
~ This Document
| The blockquote header is a \
| singular line directive.
  \end{examplesource}
  \begin{exampleoutput}
    \begin{blockquote}[This Document]
      The blockquote header is a singular line directive.
    \end{blockquote}
  \end{exampleoutput}
  \begin{examplesource}
| Unattributed text.
  \end{examplesource}
  \begin{exampleoutput}
    \begin{blockquote}
      Unattributed text.
    \end{blockquote}
  \end{exampleoutput}
\end{examples}

%%% Local Variables:
%%% mode: latex
%%% TeX-master: "0-markless"
%%% TeX-engine: luatex
%%% End:

\definesubsection{Lists}
\begin{identifier}{Ordered List}
<number ~d+> <content .*>
(<spacing ~_+> <content .*>)*
\end{identifier}
\begin{identifier}{Unordered List}
\. <content .*>
(<spacing ~_+> <content .*>)*
\end{identifier}
\definetextualcomponent{ordered list}{margin: left}
\definetextualcomponent{ordered list item}{display: list-item; list-item-prefix: number}
\definetextualcomponent{unordered list}{margin: left}
\definetextualcomponent{unordered list item}{display: list-item; list-item-prefix: dot} \\

The lists are \gpl{spanning line directive} and mark the enumeration of one or more items of a list. They can contain contain any \g{directive} with the condition that the \gpl{directive} are matched against the \g{text} of the \g{resulting textual component}. \\

After the respective list \g{identifier} has been \glink{match}{matched}, a new respective item \g{textual component} in which the higher \g{level} \g{text} is contained, is inserted for each \g{match} into the spanning \g{resulting textual component}. A single \g{match} may span over multiple \gpl{line} if the \g{text} \glink{match}{matched} by the \inline$spacing$ \g{binding} is of the same length as that of the \inline$number$ \g{binding}. In such a case, each item \g{match} itself is treated like a \g{spanning line directive} where the \g{content binding} is concatenated. \\

Ordered list items must be numbered by the \g{decimal number} given by the \inline$number$ \g{binding}, even if there is no order to how the numbers appear in the list or if there are duplicates. \\

\begin{examples}
\begin{examplesource}
. Finish this spec
. Implement a parser
\end{examplesource}
  \begin{exampleoutput}
    \begin{minipage}{0.5\textwidth}
      \begin{itemize}[noitemsep]
      \item Finish this spec
      \item Implement a parser
      \end{itemize}
    \end{minipage}
  \end{exampleoutput}
\begin{examplesource}
1 Buy some ingredients
2 Clean the kitchen
  Don't forget the sink!
5 Watch TV
\end{examplesource}
  \begin{exampleoutput}
    \begin{minipage}{0.5\textwidth}
      \begin{enumerate}[noitemsep]
      \item Buy some ingredients
      \item Clean the kitchen\\Don't forget the sink!
        \setcounter{enumi}{4}
      \item Watch TV
      \end{enumerate}
    \end{minipage}
  \end{exampleoutput}
\end{examples}

%%% Local Variables:
%%% mode: latex
%%% TeX-master: "0-markless"
%%% TeX-engine: luatex
%%% End:

% \input{4.x-directives.tex}
% \input{4.x-headers.tex}
% \input{4.x-comments.tex}
% \input{4.x-horizontal-rules.tex}
% \input{4.x-code.tex}
% \input{4.x-embeds.tex}

%%% Local Variables:
%%% mode: latex
%%% TeX-master: "0-markless"
%%% TeX-engine: luatex
%%% End:
