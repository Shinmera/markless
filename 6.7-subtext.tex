\definesubsection{Subtext}
\defineidentifier{subtext}{v<start [(]+><content ![()].*><end [)]+>}
\definetextualcomponent{subtext}{vertical-align: sub} \\

Subtext is an \g{inline directive}. It marks the \g{text} to belong to a \g{textual component} that sets the style of the text to appear smaller and below the default text line. Only the \g{text} held by the \g{content binding} is outputted to the \g{resulting textual component}. The length of the \g{text} held by the \inline$start$ and \inline$end$ \gpl{binding} must be the same in order for the \ident{subtext} \g{identifier} to \g{match}.\\

\begin{examples}
  \example{This is an example v(just so you know)}{This is an example \raisebox{-.4ex}{\scriptsize just so you know}}
  \longexample{Sometimes you have to be discreet v((or so they say (I wouldn't know))).}{Sometimes you have to be discreet \raisebox{-.4ex}{\scriptsize or so they say (I wouldn't know)}.}
\end{examples}

%%% Local Variables:
%%% mode: latex
%%% TeX-master: "0-markless"
%%% TeX-engine: luatex
%%% End: