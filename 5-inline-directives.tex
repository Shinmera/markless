\definesection{Inline Directives}
A \g{directive} is an \g{inline directive} if its identification is not bound to \gpl{line}. Unlike \gpl{line directive} therefore it can potentially be identified at any point in a string and span any length. \\

Any \g{textual component} specified by an \g{inline directive} can only contain \gpl{textual component} specified by \gpl{inline directive}. Furthermore, an \g{inline directive} cannot contain another \g{inline directive} of its own type at any \g{level}. An \g{inline directive} may further restrict which \gpl{directive} may appear within itself.

\definesubsubsection{Surrounding Inline Directives}
An \g{inline directive} is an \g{surrounding inline directive} if its \g{identifier} is terminated by a \syntax{binding-reference} to a \syntax{binding} at the start of the \g{identifier}. The \g{resulting textual component} of a \g{surrounding inline directive} never contains the \gpl{string} \glink{match}{matched} by the \g{identifier}'s starting \syntax{binding}. Instead, each \g{surrounding inline directive} \g{identifier} contains a \g{content binding} that matches all the \g{text} that the \g{resulting textual component} will contain.

\definesubsubsection{Entity Inline Directives}
An \g{inline directive} is an \g{entity inline directive} if its \g{identifier} does not contain any \syntax{binding}s and instead the \g{text} of the \g{resulting textual component} is entirely dependant on the \g{entity inline directive} specification.

\definesubsubsection{Compound Inline Directives}
An \g{inline directive} is a \g{compound inline directive} if its \g{identifier} consists of multiple \syntax{binding}s the contents of which are in some form outputted to the \g{resulting textual component}.

\definesubsection{Bold}
\defineidentifier{bold}{<start [*]+><content ![*].*><start>}
\definetextualcomponent{bold}{font-weight: bold} \\


The bold \g{directive} marks the \g{text} to belong to a \g{textual component} that sets the weight of the font to bold. Only the \g{text} held by the \inline{content} binding is outputted to the \g{resulting textual component}. \\

\begin{examples}
  \example{not *bold* at all}{not \textbf{bold} at all}
  \example{and ***some *things* are bad***}{and \textbf{some *things* are bad}}
\end{examples}

%%% Local Variables:
%%% mode: latex
%%% TeX-master: "0-markless"
%%% TeX-engine: luatex
%%% End:

\definesubsection{Italic}
\defineidentifier{italic}{<start [/]+><content ![/].*><start>}
\definetextualcomponent{italic}{font-style: italic} \\

The italic \g{directive} marks the \g{text} to belong to a \g{textual component} that sets the style of the font to italic. Only the \g{text} held by the \inline{content} binding is outputted to the \g{textual component}. \\

\begin{examples}
  \example{I /really/ don't care.}{I \textit{really} don't care.}
  \example{//de/allocate// is important.}{\textit{de/allocate} is important.}
\end{examples}

%%% Local Variables:
%%% mode: latex
%%% TeX-master: "0-markless"
%%% TeX-engine: luatex
%%% End:

\definesubsection{Underline}
\defineidentifier{underline}{<start [_]+><content ![_].*><start>}
\definetextualcomponent{underline}{text-decoration: underline} \\

The underline \g{directive} marks the \g{text} to belong to a \g{textual component} that sets the style of the text to underline. Only the \g{text} held by the \inline{content} binding is outputted to the \g{resulting textual component}. \\

\begin{examples}
  \example{We _must_ finish this.}{We \underline{must} finish this.}
  \example{This __CONSTANT_VALUE__ is variable.}{This \underline{CONSTANT\_VALUE} is variable.}
\end{examples}

%%% Local Variables:
%%% mode: latex
%%% TeX-master: "0-markless"
%%% TeX-engine: luatex
%%% End:

\definesubsection{Strikethrough}
\defineidentifier{strikethrough}{<start [-]+><content ![-].*><start>}
\definetextualcomponent{strikethrough}{text-decoration: strikethrough} \\

The strikethrough \g{directive} marks the \g{text} to belong to a \g{textual component} that sets the style of the text to strikethrough. Only the \g{text} held by the \inline{content} binding is outputted to the \g{textual component}. \\

\begin{examples}
  
\end{examples}

%%% Local Variables:
%%% mode: latex
%%% TeX-master: "0-markless"
%%% TeX-engine: luatex
%%% End:

\definesubsection{Code}
\defineidentifier{code}{<start [`]+><content ![`].*><start>}
\definetextualcomponent{code}{font-family: monospace} \\

The code \g{directive} marks the \g{text} to belong to a \g{textual component} that sets the font-family to monospace. Only the \g{text} held by the \inline{content} binding is outputted to the \g{resulting textual component}. The code \g{directive} cannot contain any other \gpl{directive}. \\

\begin{examples}
  \example{Call `compile`}{Call \inline$compile$}
  \example{Earmuffs `*around*` your specials.}{Earmuffs \inline$*around*$ your specials.}
\end{examples}

%%% Local Variables:
%%% mode: latex
%%% TeX-master: "0-markless"
%%% TeX-engine: luatex
%%% End:

\definesubsection{Dashes}
\defineidentifier{em-dash}{--}
\definetextualcomponent{em-dash}{display: em-dash} \\

If the \g{document} does not have direct for em-dashes, the unicode character \unicode{2014} may be used instead. \\

\begin{examples}
  \example{A game -- or gamble, if you will.}{A game --- or gamble, if you will.}
\end{examples}

%%% Local Variables:
%%% mode: latex
%%% TeX-master: "0-markless"
%%% TeX-engine: luatex
%%% End:

% \input{5.x-subtext.tex}
% \input{5.x-supertext.tex}
% \input{5.x-spoiler.tex}
% \input{5.x-color.tex}
% \input{5.x-size.tex}
% \input{5.x-hyperlinks.tex}
% \input{5.x-embeds.tex}

%%% Local Variables:
%%% mode: latex
%%% TeX-master: "0-markless"
%%% TeX-engine: luatex
%%% End: