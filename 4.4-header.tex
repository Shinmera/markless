\definesubsection{Header}
\begin{identifier}{header}
<level #+> <content .+>
\end{identifier}
\definetextualcomponent{header}{font-weight:bold; font-size: 1-level; indent: true; label: content} \\

The header is a \g{singular line directive}. It represents a section heading. Only the \g{text} held by the \g{content binding} is outputted to the \g{resulting textual component}. The header can only contain \gpl{inline directive}.\\

The length of the \inline$level$ \g{binding} determines the level of the heading. The level may potentially be infinitely high, though the \g{implementation} may represent levels above a certain number in the same manner. It must however support a different representation for at least levels 1 and 2. Generally, the higher the level, the smaller the font size of the heading should be. \\

An \g{implementation} may choose to number each header, where this number prefix is put together by the number prefix of the header on a level one higher followed by a dot and a counter representing how many headers of the same level have appeared until and including the current one since the last header of a higher level. In the case of a level one heading only the counter is used, as there is no higher level prefix to prepend. In the case where no level one higher is contained in the \g{document}, the level is treated as if it existed with the counter for it being 0.\\

The \g{resulting textual component} is associated with a \g{label} of the same name as the \g{text} of the \g{resulting textual component}. \\

\begin{examples}
  \begin{examplesource}
# Header
The header is a singular line
directive
## Subsection
That allows neat sectioning!
  \end{examplesource}
  \begin{exampleoutput}
    \textbf{\quad\Large Header}\\
    The header is a singular line\\
    directive.\\
    \textbf{\quad\large Subsection}\\
    That allows neat sectioning!\\
  \end{exampleoutput}
\begin{examplesource}
# Cooking a Lasagna
Here's what you have to buy:
## Ingredients
A buncha stuff!
## Steps
It's a lengthy recipe, but finally \
you'll have to
#### Bake it
\end{examplesource}
  \begin{exampleoutput}
    \textbf{\quad\Large 1 Cooking a Lasagna}\\
    Here's what you have to buy: \\
    \textbf{\quad\large 1.1 Ingredients}\\
    A buncha stuff! \\
    \textbf{\quad\large 1.2 Steps}\\
    It's a lengthy recipe, but finally you'll have to \\
    \textbf{\quad\small 1.2.0.1 Bake it}\\
  \end{exampleoutput}
\end{examples}

%%% Local Variables:
%%% mode: latex
%%% TeX-master: "0-markless"
%%% TeX-engine: luatex
%%% End: