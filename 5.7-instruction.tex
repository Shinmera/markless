\definesubsection{Instruction}
\begin{identifier}{instruction}
! <instruction .*>
\end{identifier}

The instruction is a \g{singular line directive}. Its purpose is to interact with the \g{implementation} and cause it to perform differently. There is no corresponding \g{resulting textual component} for the comment \g{directive} and as such it must not have any effect on the \g{document}. \\

The following instructions and their effect must be supported by an \g{implementation}. \\

\renewcommand{\arraystretch}{1.3}
\begin{tabularx}{\textwidth}{lX}
  \inline$set <variable> <value>$ & Sets an internal variable of the \g{implementation} to a certain value. An \g{implementation} may check the value for validity and \glink{signalling}{signal} an \g{error} if it is invalid. \\
  \inline$warn <message>$ & Causes the \g{implementation} to \glink{signalling}{signal} a \g{warning} with the given message. \\
  \inline$error <message>$ & Causes the \g{implementation} to \glink{signalling}{signal} an \g{error} with the given message. \\
  \inline$include <file>$ & Literally splices the contents of the specified file into the \g{document} in place of this instruction. The \g{implementation} must carry on to \glink{interpretation}{interpret} the newly spliced \g{text}. \\
  \inline$disable-directives <directive>*$ & Adds the named directives to the list of \gpl{disabled directive}. \\
  \inline$enable-directives <directive>*$ & Removes the named directives from the list of \gpl{disabled directive}. \\
\end{tabularx}
\renewcommand{\arraystretch}{1}

\begin{examples}
\begin{examplesource}
! set line-break-mode always
foo
bar\
baz
! set line-break-mode never
bada
boom
\end{examplesource}
\begin{exampleoutput}
foo\\
bar\\
baz\\
badaboom
\end{exampleoutput}
\begin{examplesource}
! disable-directives instruction
! error Exit!
\end{examplesource}
\begin{exampleoutput}
! error Exit!
\end{exampleoutput}
\end{examples}

%%% Local Variables:
%%% mode: latex
%%% TeX-master: "0-markless"
%%% TeX-engine: luatex
%%% End: