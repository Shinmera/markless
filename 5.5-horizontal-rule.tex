\definesubsection{Horizontal Rule}
\begin{identifier}{horizontal-rule}
==+
\end{identifier}
\definetextualcomponent{horizontal-rule}{display: line} \\

The horizontal rule is a \g{singular line directive}. It is translated into a \g{resulting textual component} that represents a horizontal rule or break on the page. This must span the entire width of the document and could be represented by a thin line. If the \g{document} cannot support the drawing of lines, the horizontal rule may instead be approximated through other means.\\

\begin{examples}
  \example{==}{\rule{0.5\textwidth}{1pt}}
  \begin{examplesource}
And now, for a brief break.
=====
Back to the show!
  \end{examplesource}
  \begin{exampleoutput}
    And now, for a brief break. \\
    \rule{0.5\textwidth}{1pt} \\
    Back to the show!
  \end{exampleoutput}
\end{examples}

%%% Local Variables:
%%% mode: latex
%%% TeX-master: "0-markless"
%%% TeX-engine: luatex
%%% End: