\definesubsubsection{Color}
\defineidentifier{compound-color}{(color (<hex #.+>|<r ~n+>,<g ~n+>,<b ~n+>))|<name .+>}
\definestyle{compound-color}{color: color} \\

If given, this option marks the \g{style} to change the colour. The colour can be given in three ways:
\begin{enumerate}
\item Through a hexadecimal notation, contained in the \inline$hex$ \g{binding}. The \g{hexadecimal number} following the \inline$#$ must be exactly six \gpl{character} long.
\item Through a red, green, blue component notation, contained in the \inline$r$,\inline$g$, and \inline$b$ \gpl{binding}. Each of these bindings must contain a \g{decimal number} that may only range between 0 and 255. If the number lies outside this range, it is clamped to the nearest boundary.
\item Through an explicit colour name, contained in the \inline$name$ \g{binding}. The name must be \g{case insensitive}. The set of supported colour names is \g{implementation dependant}.
\end{enumerate}
If the specified colour value is invalid or unknown to the \g{implementation} according to the above restrictions, no colour change occurs. If the \g{document} does not support the specified colour, the \g{implementation} must choose an alternative colour that approximates the specified one as closely as possible. \\

\definecolor{5-10-7-2}{RGB}{157,14,204}
\definecolor{5-10-7-3}{RGB}{145,16,16}
\begin{examples}
  \example{This is blue(in blue).}{This is \textcolor{blue}{blue}.}
  \example{Magic!(in color #9D0ECC)}{\textcolor{5-10-7-2}{Magic!}}
  \example{Now in technicolor(in color 145,16,16).}{Now in \textcolor{5-10-7-3}{technicolor}.}
\end{examples}

%%% Local Variables:
%%% mode: latex
%%% TeX-master: "0-markless"
%%% TeX-engine: luatex
%%% End:
