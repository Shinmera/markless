\definesubsection{Footnote-Reference}
\begin{identifier}{footnote-reference}
  \[<target ~n+>\]
\end{identifier}
\definetextualcomponent{footnote-reference}{interaction:link;target:target;vertical-align:super} \\

The footnote-reference is an \g{entity inline directive} that marks the \g{text} to belong to a \g{textual component} that sets its interaction to allow following to the \g{label} with the name held by the \g{text} of the \inline$target$ \g{binding}. The user must be presented with an action that allows them to follow to the corresponding label. The exact manner in which the target is followed as well as the way in which the action is presented are \g{implementation dependant}. The \g{text} of the \g{resulting textual component} must be exactly the same as that of the entire \g{identifier}.\\

\begin{examples}
  \begin{examplesource}
    Examples[1] are not authoritative.
    
    [1] Examples are things like this.
  \end{examplesource}
  \begin{exampleoutput}
    Examples\raisebox{.4ex}{\scriptsize \hyperref[footnote:ex2]{[1]}} are not authoritative. \\
    \rule{0.2\textwidth}{1pt} \\
    \label{footnote:ex2}1: Examples are things like this.
  \end{exampleoutput}
\end{examples}

%%% Local Variables:
%%% mode: latex
%%% TeX-master: "markless"
%%% TeX-engine: luatex
%%% End: