\definesubsubsection{Embed Parameters}
The parameters are processed in the order they are given and can effect both the content of the \g{resulting textual component} as well as its \g{style}. A parameter may also affect the processing of parameters after it. Two general types of parameters are defined: flag parameters and value parameters. Flag parameters are single parameters that add or remove an attribute from the \g{resulting textual component}'s \g{style}. Value parameters add an attribute whose value is determined by the parameter following the current one. The following parameter is then skipped over and thus not processed. \\

An \g{implementation} must at least support the parameters specified in this section if permitted by the output \g{document}, but may add additional parameters the implications of which are completely \g{implementation dependant}. If the output \g{document} does not support a particular parameter, or an unknown parameter is given, a \g{warning} is \glink{signalling}{signalled}. \\

\definesubsubsubsection{Float}
\defineidentifier{embed-property-float}{float <orientation left|right>}
\definestyle{embed-property-float}{float: orientation}

%%% Local Variables:
%%% mode: latex
%%% TeX-master: "0-markless"
%%% TeX-engine: luatex
%%% End:

\definesubsubsubsection{Width}
\defineidentifier{embed-property-width}{width <size ![ ]*>}
\definestyle{embed-property-width}{width: size}

%%% Local Variables:
%%% mode: latex
%%% TeX-master: "0-markless"
%%% TeX-engine: luatex
%%% End:

\definesubsubsubsection{Height}
\defineidentifier{embed-property-height}{height (<pixels ~n+px>|<percent ~n+\%>)}
\definestyle{embed-property-height}{height: size} \\

Causes the embed content's height to be fixed to the specified size. The size can be given in either \inline$pixels$ or \inline$percent$ where \inline$pixels$ will set the height to be the exact amount of pixels given if the document is viewed at its native resolution. \inline$percent$ will scale the height to the given percentage of the height of the \g{document}. If the \g{document} should not have a height, the \inline$percent$ specification does nothing. Unless the \ident{embed-property-width} is also specified, the embed content's aspect ratio must be preserved. \\

%%% Local Variables:
%%% mode: latex
%%% TeX-master: "0-markless"
%%% TeX-engine: luatex
%%% End:


%%% Local Variables:
%%% mode: latex
%%% TeX-master: "markless"
%%% TeX-engine: luatex
%%% TeX-command-extra-options: "-shell-escape"
%%% End:
