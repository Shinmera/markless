\definesubsection{Instruction}
\begin{identifier}{instruction}
! <instruction .*>
\end{identifier}

The instruction is a \g{singular line directive}. Its purpose is to interact with the \g{implementation} and cause it to perform differently. There is no corresponding \g{resulting textual component} for the comment \g{directive} and as such it must not have any effect on the \g{document}. \\

The following instructions and their effect must be supported by an \g{implementation}. \\
\begin{tabularx}{\textwidth}{lX}
  \inline$disable-directives <directive>*$ & Adds the named directives to the list of \gpl{disabled directive}. \\
  \inline$enable-directives <directive>*$ & Removes the named directives from the list of \gpl{disabled directive}. \\
  \inline$set-source-language <language>$ & Sets the source language by which the rest of the document should be \glink{interpretation}{interpreted}. If the \g{implementation} does not support the requested language, an \g{error} is \glink{signalling}{signalled}.\\
\end{tabularx} \\

%%% Local Variables:
%%% mode: latex
%%% TeX-master: "0-markless"
%%% TeX-engine: luatex
%%% End: