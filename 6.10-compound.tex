\definesubsection{Compound}
\defineidentifier{compound}{"<content .>"\\(<option .*>(, +<option .*>)*\\)}
\definetextualcomponent{compound}{} \\

The compound \g{directive} is a \g{compound inline directive}. It determines its \g{style} dynamically by the additive combination of present options in the \inline$option$ binding. In the case where the style combination of two options conflicts, the style of the last option has priority. \\

Only the \g{text} held by the \g{content binding} is outputted to the \g{resulting textual component}. The \inline$option$ binding cannot contain any other \gpl{directive}.\\

An \g{implementation} must at least support the options specified in this section, but may add additional options the syntax and implications of which are completely \g{implementation dependant}. If an option is found that the \g{implementation} does not support, it is ignored and a \g{warning} may be \glink{signalling}{signalled}. \\

When a compound \g{directive} is \glink{processing}{processed}, processing begins anew over the \g{content binding}. Once the \inline$options$ binding has been \glink{full match}{fully matched}, the \g{resulting textual component} is ended and control is handed back to the \g{standard processing loop}.

\definesubsubsection{Bold}
\defineidentifier{compound-bold}{bold}
\definestyle{compound-bold}{font-weight: bold} \\

If given, this option marks the \g{style} to bold the \g{text}.\\

\begin{examples}
  \example{Not "again"(in bold)!}{Not \textbf{again}!}
\end{examples}

%%% Local Variables:
%%% mode: latex
%%% TeX-master: "markless"
%%% TeX-engine: luatex
%%% TeX-command-extra-options: "-shell-escape"
%%% End:

\definesubsubsection{Italic}
\defineidentifier{compound-italic}{italic}
\definestyle{compound-italic}{font-style: italic} \\

If given, this option marks the \g{style} to italicise the \g{text}.\\

\begin{examples}
  \example{This is "really"(in italic) important!}{This is \textit{really} important!}
\end{examples}

%%% Local Variables:
%%% mode: latex
%%% TeX-master: "markless"
%%% TeX-engine: luatex
%%% TeX-command-extra-options: "-shell-escape"
%%% End:

\definesubsubsection{Underline}
\defineidentifier{compound-underline}{underline}
\definestyle{compound-underline}{text-decoration: underline} \\

If given, this option marks the \g{style} to be set to underline the \g{text}.\\

\begin{examples}
  \example{Solve it "today"(in underline)!}{Solve it \underline{today}!}
\end{examples}

%%% Local Variables:
%%% mode: latex
%%% TeX-master: "markless"
%%% TeX-engine: luatex
%%% TeX-command-extra-options: "-shell-escape"
%%% End:

\definesubsubsection{Strikethrough}
\defineidentifier{compound-strikethrough}{strikethrough}
\definestyle{compound-strikethrough}{text-decoration: strikethrough} \\

If given, this option marks the \g{style} to be set to strikethrough the \g{text}.\\

\begin{examples}
  \example{"This is a good idea"(strikethrough).}{\sout{This is a good idea}.}
\end{examples}

%%% Local Variables:
%%% mode: latex
%%% TeX-master: "markless"
%%% TeX-engine: luatex
%%% TeX-command-extra-options: "-shell-escape"
%%% End:

\definesubsubsection{Spoiler}
\defineidentifier{compound-spoiler}{spoiler}
\definestyle{compound-spoiler}{display: hidden} \\

If given, this option marks the \g{style} to obscure the \g{text} in such a manner that the \g{user} must perform an \g{action} in order to reveal the \g{text}.\\

\begin{examples}
  \example{This is a secret(in spoiler)!}{This is a \colorbox{black}{secret}!}
\end{examples}

%%% Local Variables:
%%% mode: latex
%%% TeX-master: "0-markless"
%%% TeX-engine: luatex
%%% End:

\definesubsubsection{Font}
\defineidentifier{compound-font}{font <font .+>}
\definestyle{compound-font}{font-family: font} \\

If given, this option marks the \g{style} to change the font family. If the specified font is not available to the \g{user} for one reason or another, no font change occurs. The \g{implementation} may make an effort to include the font in the \g{document} in such a way that it is not necessary for the user to have a copy of the font, but it is not required to. \\

\begin{examples}
  \longexample{"Comic sans"(in font Comic Sans Ms) is a good font to annoy people.}{{\fontspec{Comic Sans MS}\selectfont Comic sans} is a good font to annoy people.}
\end{examples}

%%% Local Variables:
%%% mode: latex
%%% TeX-master: "markless"
%%% TeX-engine: luatex
%%% End:

\definesubsubsection{Color}
\defineidentifier{compound-color}{(color (<hex #.+>|<r ~n+>,<g ~n+>,<b ~n+>))|<name .+>}
\definestyle{compound-color}{color: color} \\

If given, this option marks the \g{style} to change the colour. The colour can be given in three ways:
\begin{enumerate}
\item Through a hexadecimal notation, contained in the \inline$hex$ \g{binding}. The \g{hexadecimal number} following the \inline$#$ must be exactly six \gpl{character} long.
\item Through a red, green, blue component notation, contained in the \inline$r$,\inline$g$, and \inline$b$ \gpl{binding}. Each of these bindings must contain a \g{decimal number} that may only range between 0 and 255. If the number lies outside this range, it is clamped to the nearest boundary.
\item Through an explicit colour name, contained in the \inline$name$ \g{binding}. The name must be \g{case insensitive}. The set of supported colour names is \g{implementation dependant}.
\end{enumerate}
If the specified colour value is invalid or unknown to the \g{implementation} according to the above restrictions, an \g{error} is \glink{signalling}{signalled}. If the \g{document} does not support the specified colour, the \g{implementation} must choose an alternative colour that approximates the specified one as closely as possible. \\

\definecolor{5-10-7-2}{RGB}{157,14,204}
\definecolor{5-10-7-3}{RGB}{145,16,16}
\begin{examples}
  \example{This is blue(in blue).}{This is \textcolor{blue}{blue}.}
  \example{Magic!(in color #9D0ECC)}{\textcolor{5-10-7-2}{Magic!}}
  \example{Now in technicolor(in color 145,16,16).}{Now in \textcolor{5-10-7-3}{technicolor}.}
\end{examples}

%%% Local Variables:
%%% mode: latex
%%% TeX-master: "markless"
%%% TeX-engine: luatex
%%% End:

\definesubsubsection{Size}
\defineidentifier{compound-size}{(size )?(<point ~n+pt>|<em ~n+?(\\.~n+?)?em>|<name .+>)}
\definestyle{compound-size}{font-size: size} \\

This option marks the \g{style} to change the font size. The size can be given in three ways:
\begin{enumerate}
\item Through a point value, contained in the \inline$point$ \g{binding}. The \g{real number} must be greater than zero.
\item Through an em value, contained in the \inline$em$ \g{binding}. The \g{real number} must be greater than zero. The font size is scaled according to the \g{real number} multiplied by the font size of the \g{textual component} one \g{level} below.
\item Through a name, contained in the \inline$name$ \g{binding}. The name must be \g{case insensitive}. At least the following names, corresponding to scaling factors, must be supported by the \g{implementation}:
  \begin{itemize}[noitemsep]
  \item Microscopic 0.25em
  \item Tiny 0.5em
  \item Small 0.8em
  \item Normal 1.0em
  \item Big 1.5em
  \item Large 2.0em
  \item Huge 2.5em
  \item Gigantic 4.0em
  \end{itemize}
  An implementation may support additional names, the exact sizing effects of which are \g{implementation dependant}.
\end{enumerate}
If the specified size value is invalid or unknown to the \g{implementation} according to the above restrictions, no size change occurs. \\

\begin{examples}
  \example{Oh "shit!"(in huge)}{Oh {\fontsize{2.5em}{2.6em}\selectfont shit!}}
  \example{In "20pt."(in 20pt)}{In {\fontsize{20pt}{20pt}\selectfont 20pt.}}
  \longexample{Well ""uh, "I don't know..."(in size 0.5em)""(in size 0.8em)}{Well {\fontsize{0.8em}{0.8em}\selectfont uh, {\fontsize{0.5em}{0.5em}\selectfont I don't know...}}}
\end{examples}

%%% Local Variables:
%%% mode: latex
%%% TeX-master: "0-markless"
%%% TeX-engine: luatex
%%% End:

\definesubsubsection{Hyperlink}
\defineidentifier{compound-hyperlink}{\{url\}|(#<internal .+>)|(link <external .+>)}
\definestyle{compound-hyperlink}{interaction: link;target: target} \\

This option marks the \g{style} to set the interaction to allow following to the target. The user must be presented with an action that allows them to follow to the target. The exact manner in which the target is followed as well as the way in which the action is presented are \g{implementation dependant}. The target can be given in three ways:
\begin{enumerate}
\item As an URL, contained in the \inline$target$ \g{binding}. In this case the semantics are the same as for the \comp{URL} \g{textual component}.
\item As an external reference, contained in the \inline$external$ \g{binding}. The exact semantics and allowed values for external references are \g{implementation dependant}.
\item As an internal reference, contained in the \inline$internal$ \g{binding}. The target is set to the position of the \g{textual component} associated with the \g{label} of the same name as the contents of the \g{binding}.
\end{enumerate}
If the specified target is invalid or unknown to the \g{implementation} according to the above restrictions, no interaction change occurs. \\

\begin{examples}
  \longexample{The "hyperspec"(to http://l1sp.org/cl/) is very useful.}{The \href{http://l1sp.org/cl/}{hyperspec} is very useful.}
  \example{And in "part 2"(to #identifier-syntax)...}{And in \hyperref[section:IDENTIFIER SYNTAX]{part 2}...}
  \example{I drew "something"(to ~/drawings/test.jpg) today.}{I drew \href{run:~/drawings/test.jpg}{something} today.}
\end{examples}

%%% Local Variables:
%%% mode: latex
%%% TeX-master: "markless"
%%% TeX-engine: luatex
%%% TeX-command-extra-options: "-shell-escape"
%%% End:


%%% Local Variables:
%%% mode: latex
%%% TeX-master: "markless"
%%% TeX-engine: luatex
%%% TeX-command-extra-options: "-shell-escape"
%%% End: