\definesubsection{Strikethrough}
\defineidentifier{strikethrough}{<start \\<-+><content ![-].*><end -+\\>>}
\definetextualcomponent{strikethrough}{text-decoration: strikethrough} \\

Strikethrough is a \g{surrounding inline directive}. It marks the \g{text} to belong to a \g{textual component} that sets the style of the text to strikethrough. Only the \g{text} held by the \g{content binding} is outputted to the \g{resulting textual component}. The length of the \g{text} held by the \inline$start$ and \inline$end$ \gpl{binding} must be the same in order for the \ident{strikethrough} \g{identifier} to \g{match}.\\

\begin{examples}
  \example{To Do: <-nothing->}{To Do: \sout{nothing}}
  \example{<-Solve LOAD-TIME-VALUE problem->}{\sout{Solve LOAD-TIME-VALUE problem}}
\end{examples}

%%% Local Variables:
%%% mode: latex
%%% TeX-master: "0-markless"
%%% TeX-engine: luatex
%%% End:
