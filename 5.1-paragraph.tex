\definesubsection{Paragraph}
\begin{identifier}{paragraph}
<spaces [ ]*><content![ ].*>
\end{identifier}
\definetextualcomponent{paragraph}{margin: top, bottom} \\

The paragraph can only be \glink{match}{matched} if no other \g{line directive} \glink{match}{matches}. \Gpl{line} belong to the same paragraph until the length of \inline{spaces} changes, a new \g{inline directive} is recognised, or an \g{empty line} is encountered. The paragraph is a \g{spanning line directive}. \\

Paragraphs are visually distinguished by a margin above and below the \g{text}. An \g{implementation} may additionally employ indentation rules to distinguish the beginning of a paragraph. \\

\begin{examples}
  \begin{examplesource}
This is a paragraph
that spans multiple lines

This is another paragraph.
  \end{examplesource}
  \begin{exampleoutput}
    This is a paragraph\\
    that spans multiple lines.\\
    \\
    This is another paragraph.
  \end{exampleoutput}
  \begin{examplesource}
Paragraph One
  Paragraph Two
  \end{examplesource}
  \begin{exampleoutput}
    Paragaph One\\
    \\
    Paragraph Two
  \end{exampleoutput}
\end{examples}

%%% Local Variables:
%%% mode: latex
%%% TeX-master: "markless"
%%% TeX-engine: luatex
%%% TeX-command-extra-options: "-shell-escape"
%%% End:
